\documentclass[]{scrreprt}


% Title Page
\title{Change Management}
\author{David Diks, Jandie Hendriks, Yannick Riesenbos}


\begin{document}
\maketitle

\chapter{Terminologie}

\section{Change Request}

	Een change-request is een aanvraag voor het veranderen van een functie of het implementeren van nieuwe functies. Hierbij wordt de gevraagde functionaliteit in redelijk detail beschreven, inclusief een business-case voor de functie. Het formaat hiervan zal daarom hetzelfde zijn als use-cases.
	
	Een request is geen commitment maar zal nog beoordeeld worden door de verschillende rollen.

	\section{Rollen}
	De rollen zijn de verschillende stakeholders binnen het project. Deze zijn als volgt.
	
	\subsection{Requester}
	
	Deze rol is voor degene die een Change Request invoert. Hij of zij wil iets nieuws zien in de applicatie
	
	\subsection{Architect}
	
	De architect heeft zicht op de applicatie en weet wat wel en niet te implementeren is.
	
	\subsection{Beheer}
	
	Beheer overziet de test- en productie omgevingen. Hij of zij kan aangeven of bepaalde veranderingen wel zijn toegestaan. Dit is vaak de  CI-verantwoordelijke.
	
	\subsection{Management}
	
	De manager beheert het process en besluit of er tijd is voor specifieke changes.
	
	\subsection{Developer}
	
	Developers implementeren de features en hebben een bepaald inzicht die andere rollen missen.
	
	\chapter{Change Requests}
	
	Een change-request doorloopt de volgende stappen:
	
	\begin{enumerate}
		\item Report
		\item Estimation
		\item Review
		\item Implementation
		\item Deployment
	\end{enumerate}

	\section{Report}
	
	Roles: \textit{Requester} \\
	
	In deze fase wordt een change-request aangemaakt. Deze volgt het volgende formaat: 
	
	\begin{quotation}
		as a \textbf{Role} I want to \textbf{Action} because \textbf{Business value}.
	\end{quotation}
	
	\section{Estimation}
	
	Rollen: \textit{Architect, Developer} \\
	
	Tijdens estimation worden er inschattingen gemaakt over de haalbaarheid van een change-request. Er wordt gekeken of deze binnen de bestaande architectuur past en of deze uitvoerbaar is. \\
	
	Hierbij is het mogelijk dat er problemen voorkomen, zoals;
	
	\begin{itemize}
		\item De functie vergt te veel rekenkracht
		\item De functie past niet binnen het doel
		\item De functie past niet binnen voorgezette doelen voor architectuur
	\end{itemize}

	Indien de change-request goedgekeurd wordt gaat deze door naar Review. Indien niet, zal de request afgesloten worden.

	\section{Review}
	
	Rollen: \textit{Management, Developer} \\
	
	In deze fase wordt bepaald of de veranderingen te implementeren zijn en of er tijd voor is.
	
	Mogelijke problemen zijn:
	
	\begin{itemize}
		\item Er is geen tijd voor
		\item Er is geen geld voor
		\item De use-case is niet nauw genoeg
		\item De implementatie gaat lang duren
	\end{itemize}
	
	Indien er geen fouten zijn gaat de request door naar implementatie, anders wordt deze afgesloten met relevant kommentaar. Indien de request hier geaccepteerd wordt zal deze geimplementeerd worden.
	
	\section{Implementatie}
	
	Rollen: \textit{Developer} \\
	
	In deze fase wordt de code ontwikkeld, getest en gereviewed. Zie hiervoor ook het OTAP-document. Code gaat automatisch door naar deployment
	
	\section{Deployment}
	
	Rollen: \textit{Beheer} \\
	
	In deze fase word de code gedeployed en de change request afgesloten.
\end{document}          
